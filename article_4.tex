%%%%%%%%%%%%%%%%%%%%%%%%%%%%%%%%%%%%%%%%%
% Arsclassica Article
% LaTeX Template
% Version 1.1 (1/8/17)
%
% This template has been downloaded from:
% http://www.LaTeXTemplates.com
%
% Original author:
% Lorenzo Pantieri (http://www.lorenzopantieri.net) with extensive modifications by:
% Vel (vel@latextemplates.com)
%
% License:
% CC BY-NC-SA 3.0 (http://creativecommons.org/licenses/by-nc-sa/3.0/)
%
%%%%%%%%%%%%%%%%%%%%%%%%%%%%%%%%%%%%%%%%%

%----------------------------------------------------------------------------------------
%	PACKAGES AND OTHER DOCUMENT CONFIGURATIONS
%----------------------------------------------------------------------------------------

\documentclass[
10pt, % Main document font size
a4paper, % Paper type, use 'letterpaper' for US Letter paper
oneside, % One page layout (no page indentation)
%twoside, % Two page layout (page indentation for binding and different headers)
headinclude,footinclude, % Extra spacing for the header and footer
BCOR5mm, % Binding correction
]{scrartcl}


\input{structure.tex} % Include the structure.tex file which specified the document structure and layout

\hyphenation{Fortran hy-phen-ation} % Specify custom hyphenation points in words with dashes where you would like hyphenation to occur, or alternatively, don't put any dashes in a word to stop hyphenation altogether

%----------------------------------------------------------------------------------------
%	TITLE AND AUTHOR(S)
%----------------------------------------------------------------------------------------

\title{\normalfont\spacedallcaps{Brittle fracture in Charpy impact test}} % The article title
\subtitle{\normalfont\spacedallcaps{Influence of material parameters}}

%\subtitle{Subtitle} % Uncomment to display a subtitle

\author{\spacedlowsmallcaps{Jules Royer}} % The article author(s) - author affiliations need to be specified in the AUTHOR AFFILIATIONS block

\date{\today} % An optional date to appear under the author(s)

%----------------------------------------------------------------------------------------

\begin{document}

%----------------------------------------------------------------------------------------
%	HEADERS
%----------------------------------------------------------------------------------------

\renewcommand{\sectionmark}[1]{\markright{\spacedlowsmallcaps{#1}}} % The header for all pages (oneside) or for even pages (twoside)
%\renewcommand{\subsectionmark}[1]{\markright{\thesubsection~#1}} % Uncomment when using the twoside option - this modifies the header on odd pages
\lehead{\mbox{\llap{\small\thepage\kern1em\color{halfgray} \vline}\color{halfgray}\hspace{0.5em}\rightmark\hfil}} % The header style

\pagestyle{scrheadings} % Enable the headers specified in this block

%----------------------------------------------------------------------------------------
%	TABLE OF CONTENTS & LISTS OF FIGURES AND TABLES
%----------------------------------------------------------------------------------------

\maketitle % Print the title/author/date block

\setcounter{tocdepth}{2} % Set the depth of the table of contents to show sections and subsections only

\tableofcontents % Print the table of contents

% \listoffigures % Print the list of figures

% \listoftables % Print the list of tables

%----------------------------------------------------------------------------------------
%	ABSTRACT
%----------------------------------------------------------------------------------------

%----------------------------------------------------------------------------------------
%	AUTHOR AFFILIATIONS
%----------------------------------------------------------------------------------------

% \let\thefootnote\relax\footnotetext{* \textit{Department of Biology, University of Examples, London, United Kingdom}}

% \let\thefootnote\relax\footnotetext{\textsuperscript{1} \textit{Department of Chemistry, University of Examples, London, United Kingdom}}

%----------------------------------------------------------------------------------------

\newpage % Start the article content on the second page, remove this if you have a longer abstract that goes onto the second page

%----------------------------------------------------------------------------------------
%	INTRODUCTION
%----------------------------------------------------------------------------------------

\section{Introduction}
Goal of this work is to analyze the conditions under which fractures
and cracks propagate in materials. It relies on the theory of fracture mechanics.
It has many important applications, especially in the industry field,
in which built mechanical components often have flaws.
The probability of the failure of the material during
its operation must be assessed.
To that end, engineers do a damage tolerance analysis.
The Charpy impact test is commonly used to study fractures.
Its functioning is quite simple: a striker is dropped and hits a
notched tensile, the difference between the final and initial heights
correspond to the energy dissipated in the specimen to create a fracture.

In this work, Finite Element Element simulations of this
experiment are done to analyze
the influence of some parameters of the material on its resistance
to fracture.

\begin{figure}[H]
    \centering
    \includegraphics[width=0.7\columnwidth]{mouton_charpy.png}
    \caption{Charpy impact machine picture and a sketch explaining how it works.}
\end{figure}

\section{Preliminary Finite Element Model}

We present here a FEM model of the Charpy specimen.
Here it is a tensile with dimensions $L=55 \text{mm}, W = 10 \text{mm}, H = 10 \text{mm}$.
It has an
isotropic elastic behavior under low strain or stress, with a
Young Modulus $E=208 \text{GPa}$ which is standard for steel.
It then has a isotropic non linear plastic behavior ruled by the following yield
function:
$ f(\sigma, p) = \sigma_{eq} - R_0 - Q(1 - \exp^{-bp}) $ where $p$ is the cumulative
plastic deformation $\displaystyle p = \int_{t=0}^{t=t*}\varepsilon_{p}(t)dt$,
and the fitted parameters are $R_0=510 \text{MPa}, Q = 314.62 \text{MPa}, b = 16.13$.

To model the plastic regime, the \textit{Von Mises} equivalent stress
is used:

\begin{equation}
    \sigma_{eq} = \sqrt{\frac{1}{2}\left((\sigma_1 - \sigma_2)^2+(\sigma_2 - \sigma_3)^2+(\sigma_3 - \sigma_1)^2\right)}
\end{equation}

Where $\sigma_i$ are the principal stresses.
Lastly, the Beremin model is used to predict the evolution of the crack.
Concerning the boundary conditions, anvils on which the specimen lies
are immobile, the striker has a prescribed vertical downward movement,
also called loas, at
constant speed $\dot{u}_2 = -1 mm.s^{-1}$. The ligament, ie the nodes
of the FEM mesh in the vertical plane cutting the specimen in half, are also immobile
along the horizontal axis.
While this assumption is not physically justified, it is useful to
shrink the problem's size to only the right half of all aforementioned
objects, because the problem can be considered symmetrical along the ligament.


\begin{figure}[ht]
    \centering
    \subfloat[Initial mesh]{\includegraphics[width=0.45\textwidth]{p1_init_mesh.png}}
    \hfill
    \subfloat[Vertical displacement]{\includegraphics[width=0.45\textwidth]{p1_u2_map.png}}
    \\
    \subfloat[\textit{Von Mises} stress]{\includegraphics[width=0.45\textwidth]{p1_sigmises_map.png}}
    \hfill
    \subfloat[cumulative plastic deformation]{\includegraphics[width=0.45\textwidth]{p1_epcum_map.png}}
    \caption{Results of experiment 1}
    \label{fig:res_exp_1}
\end{figure}

The first simulation returned an error during the last iterations.
We observe a concentration of high \textit{Von Mises} stress
around the notch and near the striker, $\sigma_{eq} > 700 \text{MPa}$, and
also a bit of
plastic deformation.
A possible explanation of the error could be a lacking mesh precision
around those zones and also near the anvils where it reaches
$\sim 310 \text{MPa}$.

\section{Local Mesh Refinement}

Thanks to this analysis, we choose a finer mesh in those areas :

\begin{figure}[h]
    \centering
    \includegraphics[width=\textwidth]{p2_refine_props.png}
    \caption{Propositions of mesh refinement.}
\end{figure}

The $4$-th mesh fulfills our desires of refinement, so we select it.
In order to select the size of the smallest mesh elements, we rerun the
previous experiment with $40 \mu m, 10 \mu m, 4 \mu m$:

\begin{figure}[ht]
    \centering
    \subfloat[Refined mesh]{\includegraphics[width=0.45\textwidth]{p2_init_mesh.png}}
    \hfill
    \subfloat[Relation force on striker - displacement]{\includegraphics[width=0.45\textwidth]{p2_force_ud.png}\label{fig:force-displ}}
    \\
    \subfloat[Relation first principal stress - y coordinate]{\includegraphics[width=0.45\textwidth]{p2_sig1_y.png}\label{fig:firstprin-y}}
    \hfill
    \subfloat[Relation cumulative plastic deformation - y coordinate]{\includegraphics[width=0.45\textwidth]{p2_epcum_y.png}\label{fig:epcum-y}}
    \caption{Comparison of mesh refinements.}
    \label{fig:res_exp_refine}
\end{figure}

With these finer meshes, the previous simulation now converges.
Results on Figures~\ref{fig:force-displ} and ~\ref{fig:firstprin-y} quite differ between $40 \mu m$
and the $2$ smaller sizes, meaning some model accuracy is gained with refinement.
However, $10 \mu m$ and $4 \mu m$ have similar results for the desired precision,
but the $10 \mu m$ simulation was $15$ times faster, therefore it is
more interesting for us to choose it instead of the smallest size.
We see again on Figures~\ref{fig:firstprin-y} and ~\ref{fig:epcum-y} the concentration
of stress and plastic deformation near the notch ie in the low $y$ zones.

\section{Geometric Parameter Sensitivity}

\begin{wrapfigure}{r}{0.5\textwidth}
    \centering
    \includegraphics[width=0.6\textwidth]{p3_geom_params.png}
    \caption{Considered geometric parameters.}
\end{wrapfigure}

Now that a more adapted mesh is found and simulations converge,
it is interesting to assess the dependency of the geometry of the Charpy specimen
in its probability of failure. Numerical simulation is here useful
to run multiple Charpy tests, that in reality are hard to set up.
The considered parameters are the notch radius $N_R$, the height $H$
and half of the width $H_W$:

The strategy is here to run $3$ simulations for each parameter,
in each of them we add a certain variation of the parameter: $-20\%$, $0\%$ and $+20\%$,
while keeping every other at its default value.

\begin{figure}[H]
    \centering
    \subfloat[Probability of failure for various $N_R$]{\includegraphics[width=0.45\textwidth]{p3_varia_NR.png}}
    \hfill
    \subfloat[Zoom]{\includegraphics[width=0.45\textwidth]{p3_varia_NR2.png}}
    \\
    \subfloat[Probability of failure for various $H$]{\includegraphics[width=0.45\textwidth]{p3_varia_H.png}}
    \hfill
    \subfloat[Zoom]{\includegraphics[width=0.45\textwidth]{p3_varia_H2.png}}
    \\
    \subfloat[Probability of failure for various $H_W$]{\includegraphics[width=0.45\textwidth]{p3_varia_Hw.png}}
    \hfill
    \subfloat[Zoom]{\includegraphics[width=0.45\textwidth]{p3_varia_Hw2.png}}

    \caption{Probability of failure with respect to the striker displacement for
        various values of $N_r$, $H$ and $H_W$.}
    \label{fig:param-var}
\end{figure}

On Figure~\ref{fig:param-var}, we see that the probability of failure curve
has a similar shape for each parameter. The plots of the right column are zooms
on the zones of highest variance between curves, with $U_d \in [0.5 \text{mm}, 0.7 \text{mm}]$.
On these zooms, all $P_f$ curves are approximately linear, with same slope
$\frac{\Delta P_f}{\Delta U_d} \sim \frac{72-48}{0.7-0.5} = 120 \%.mm^{-1}$.
Only the offsets at a given displacement $u_d$ differ between parameters variations $v_1,v_2 \in \{\pm 20\%, 0\%\}$ ie
$|P_{f,v_1}(u_d) - P_{f,v_2}(u_d)|$.
Variating $H_W$ by $\pm 20\%$ induces the smallest variation of $P_f$, of $\sim 1\%$,
where for $N_R$ and $H$, $\Delta P_f \sim 5\%$.
Therefore $H$ and $N_R$ are the 2 most impactful parameters on the failure probability.
The lower the notch radius and the higher the height, the higher the failure
probability.
Such a result can be misleading at first sight, but recall that
$N_R$ is measured at a fixed height, and therefore
a smaller $N_R$ is linked to a sharper angle at the origin of the notch.
Consequently, sharp geometries tend to concentrate the stress on them and therefore
increase the probability of failure.
For $H$, a possible interpretation is the increase of the population
of defects with the height.
But if one industrial operator should choose to focus only
on one parameter, $H$ seems simpler to
adjust than $N_R$, eventhough the latter seem more dangerous, because
the failure probability increases as $N_R$ diminishes, which makes
the notch harder to spot.

\section{The Beremin Model}

A difficulty linked to estimating the failure probability is that failure
sometimes appear eventhough the applied stress is below the yield value
of the material, because there were defects and cracks in it at a
microscopic scale.
The Beremin model focuses on linking a macroscopic fracture to those micromechanisms.

The idea of the Beremin model relies on $2$ main assumptions.
The first states that microcracks appear due to a plastic
deformation in grains, that is inhomogeneous because of initial defects.
The second models the propagation of those microcracks with a
probabilistic failure parameter, $\sigma_w$ exceeding a critical value.
To infer a macroscopic fracture from the microscopic state,
Beremin's model relies on the weakest link theory stating that the latter
occurs when a microcracks is submitted to the Griffith stress $\sigma_c$.
Equivalently, the fracture occurs when a microcrack reaches a length of
$l_{0c} = 2E\gamma_s /(\delta\sigma_c^2)$ where $\delta \sim 1$ is constant
and $\gamma_s$ is the effective surface energy.
As mathematical tools, Beremin relies on a generalized extremum law
of probability called the Weibull distribution.
For the initial distribution $\mathbb{P}_{crack}$ of defects,
a power law is commonly used, leading to the probability of
failure of an element of volume $V_0$ under a stress $\sigma$:

\begin{equation} \label{eq:pfail}
    \mathbb{P}_{V_0,fail}(\sigma) = \displaystyle \int_{l_{0c}}^{+\infty} \mathbb{P}_{crack}(l_0)d0
    = \int_{l_{0c}}^{+\infty} \frac{\alpha}{l_0^{\beta}}dl_0
    % = \left(\frac{\sigma}{\sigma_u}\right)^m
\end{equation}

Where $\alpha,\beta$ are material constraints.
By only considering plane strain condition, the probability of
failure is calculated with:

\begin{equation}
    P_{f}
    = 1 - \exp\left(- \frac{ \displaystyle  \frac{1}{V_0} \int_{V_p} \sigma_{I}^m dV }{\sigma_u^m}\right)
    = 1 - \exp\left(-\left(\frac{\sigma_w}{\sigma_u}\right)^m\right)
\end{equation}

Where $m = 2\beta - 1$ is the Weibull shape factor, $\sigma_w = \displaystyle \left( \frac{1}{V_0} \int_{V_p} \sigma_{I,p}^m \right)^{\frac{1}{m}}$
is the Weibull stress, $V_p$ the plastic volume, $\sigma_I(V)$ the principal
stress on the local volume $V$
and $\sigma_u = (m/2\alpha)^{\frac{1}{m}}\sqrt{2E\gamma_s\delta}$
is the Weibull scale factor.


% The local approach is to discretize the stressed volume $V_p$ that
% becomes plastic into $n$ smaller ones of volume $V_0$, mutually independant
% and each containing a notch.
% It is assumed the general stressed region fails if and
% only if one of its subregion fails, and the failure will appear
% at the longest microcrack.
% Given the probability distribution of the crack within one small volume :$\mathbb{P}_{crack}$,
% the probability of failure under stress $\sigma$ in this volume is the probability to find a
% crack longer than a critical length $l_c$ linked to a critical stress:

% Where $m$ and $\sigma_u$ are constant. we then find the overall
% probability of failure:
% \begin{equation}
%     P_{f} = 1 - \exp\left(-\left(\frac{\sigma_w}{\sigma_u}\right)^m\right)
% \end{equation}

% represents a mean local stress among all subvolumes.

The $2$ Beremin parameters $m$ and $\sigma_u$
have specific influence on $P_f$:
$m$ is linked to the distribution of length of microcracks (Eq~\ref{eq:pfail}).
Its increase favors smaller lengths of defects (with $\alpha^\frac{1}{m}$ as reference of length).
% Beremin originally
% used $ \mathbb{P}_{crac}(l_0)dl_0 = \frac{\alpha}{l_0^{\beta}}$ with
% $\beta = \frac{m}{2} + 1$. 
To assess its impact on $P_f$, we can
assume $\sigma_w$ is approximately constant when $m$ varies, then
an increase of $m$ increases $P_f$ if $\sigma_w > \sigma_u$ and
decreases it if $\sigma_w < \sigma_u$. For $\sigma_u$, its increase
induces a decrease of $P_f$,
it is a kind of a macroscopic stress threshold under which
the material can resist.

% Those relations are intuitive if we interpret $m$ as the number of
% subzones of defects in the stressed zone, and $\sigma_u$ a stress linked
% to the critical $\sigma_c$: increasing this threshold is equivalent
% to say the material is more resistant to fracture, where increasing
% the number of defects increases its instability, and the probability
% of failure when the local stress exceeds the critical threshold.

% ~\ref{fig:var-m-sigu}
Looking at the results of simulations on the Figure below, the trend that $m$
shifts the curve of $P_f$ to the right is visible, without noticeably changing
the slope, probably because the variation of $\sigma_w$ w.r.t $m$ must
be considered. As expected, a $\sigma_u$ reduces the slope in the
linear part.

\begin{figure}[H]
    \centering
    \subfloat[Probability of failure for various $m$.]{\includegraphics[width=0.45\textwidth]{p4_pf_ud_m.png}}
    \hfill
    \subfloat[Probability of failure for various $\sigma_u$.]{\includegraphics[width=0.45\textwidth]{p4_pf_ud_sigu.png}}
    \label{fig:var-m-sigu}
\end{figure}


We wish now to find $m$ and $\sigma_u$ for a given material, for which
we experimentally statistics of failure $P_f$ for load $u_d$.
Because the FEM model has long computation time, a regression approach
will be very time consuming.
In fact, one simulation that computes
$u_d,P_f$ given $m,\sigma_u$ takes $\sim 1$ minute.
Therefore it is preferable to adjust manually those parametres, knowing
their influence as explained before. Pretty quickly a satisfying
estimation has been found on Figure~\ref{fig:estim-m-sigu}.

The mean square error of the estimation interpolated on the experimental
values of $u_d$ is $\displaystyle \sum_{u_{d,exp}}\left(P_{f,exp}(u) - P_{f,sim}(u)\right)^2 = 10.14$.
An optimisation algorithm that minimized the mean square error gave as solutions $m_{opt}=16$ and $\sigma_{u,opt}=1800$
and an error $e_{opt} \sim 4$, so quite lower as the previous estimation.

\begin{figure}[H]
    \centering
    \includegraphics[width=0.45\textwidth]{p4_m_sig_estim2.png}
    \caption{Observed $P_f(u_d)$ (blue dots). Estimated $P_f(u_d)$ with FEM method (blue curve) and $m=15,\sigma_u=1830$.}
    \label{fig:estim-m-sigu}
\end{figure}

\section{Effect of The Temperature}

An interesting question that occurs frequently in manufacturing is the influence of the temperature on the
failure probability. Multiples simulation have been conducted to estimate the variation of $\sigma_w$ and $P_f$
with different temperatures. Results are in Figure~\ref{fig:estim-simw-temp}.

\begin{figure}[H]
    \centering
    \includegraphics[width=\columnwidth]{p5_temp_sigu_m.png}
    \caption{Variations of $\sigma_w$ and $P_f$
        with different temperatures}
    \label{fig:estim-simw-temp}
\end{figure}

The curves of $u_d \rightarrow \sigma_w(u_d)$ have same linear slope
near the origin, and different asymptotic straight lines. As $T$
decreases, the offset and the slope of those lines increases,
meaning that it increases the local stress on the plastic volume.
A possible interpretation is that lowering $T$ slow down the particles'
movements and therefore hinders the material's deformation in response
to the applied stress, as it makes the material more brittle. Similarly, lowering $T$ shift the curve of
$P_f$ to the left and increases the slope, meaning it increases the
chances of failure for all load displacements $u_d$.

\section{Conclusion}

In this work, a Finite Element Model of the Charpy impact test
has been used in order to assess the influence of various parameters on the resistance
of a notch material, ie its ability not to fracture when it is subject
to a stress near the notch. Firstly the mesh near
the highly stressed zones had te be refined in order to have convergent simulations,
and then a sensitivity analysis of $3$ geometrical parameters:
$N_R$ the notch radius, $H$ the height of the specimen and $H_W$ half
of its width, on the probability of failure $P_f$ has been conducted :
. As a result, both $N_R$ and $H$ had a significant
influence on the failure probability, and therefore should be
watched in an industrial context.
The study has then assessed the effect of the $2$ main parameters of
the Beremin Model, a local probabilitic approach to fracture model that links the fracture of the
material to the lengths of its microcracks.
CCL Beremin.

Lastly the increase of $P_f$ with a lowering of the temperature has been
observed on simulations and the induced hindering on the
movements of particles has been given as an interpretation.

As a sequel of this work, a study on the influence on the resistance
of context-specific environmental
properties to which the material is exposed can be interesting, especially
on the irridiation levels, that typically occur in nuclear infrastructures.
Multiple studies focus on this issue and link it
to the material's composition. As an example, levels of copper,
nickel and phosphorus matter when one studies the irradiation effects.

One limitation of the Beremin Model used here is its inability to model
inhomogeneities in the material. To that end, an extended version,
called the bi-modal Weibull distribution has been developped.


%----------------------------------------------------------------------------------------
%	METHODS
%----------------------------------------------------------------------------------------

%----------------------------------------------------------------------------------------
%	BIBLIOGRAPHY
%----------------------------------------------------------------------------------------

% \renewcommand{\refname}{\spacedlowsmallcaps{References}} % For modifying the bibliography heading

% \bibliographystyle{unsrt}

% \bibliography{sample.bib} % The file containing the bibliography

%----------------------------------------------------------------------------------------

\end{document}
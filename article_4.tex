%%%%%%%%%%%%%%%%%%%%%%%%%%%%%%%%%%%%%%%%%
% Arsclassica Article
% LaTeX Template
% Version 1.1 (1/8/17)
%
% This template has been downloaded from:
% http://www.LaTeXTemplates.com
%
% Original author:
% Lorenzo Pantieri (http://www.lorenzopantieri.net) with extensive modifications by:
% Vel (vel@latextemplates.com)
%
% License:
% CC BY-NC-SA 3.0 (http://creativecommons.org/licenses/by-nc-sa/3.0/)
%
%%%%%%%%%%%%%%%%%%%%%%%%%%%%%%%%%%%%%%%%%

%----------------------------------------------------------------------------------------
%	PACKAGES AND OTHER DOCUMENT CONFIGURATIONS
%----------------------------------------------------------------------------------------

\documentclass[
10pt, % Main document font size
a4paper, % Paper type, use 'letterpaper' for US Letter paper
oneside, % One page layout (no page indentation)
%twoside, % Two page layout (page indentation for binding and different headers)
headinclude,footinclude, % Extra spacing for the header and footer
BCOR5mm, % Binding correction
]{scrartcl}


\input{structure.tex} % Include the structure.tex file which specified the document structure and layout

\hyphenation{Fortran hy-phen-ation} % Specify custom hyphenation points in words with dashes where you would like hyphenation to occur, or alternatively, don't put any dashes in a word to stop hyphenation altogether

%----------------------------------------------------------------------------------------
%	TITLE AND AUTHOR(S)
%----------------------------------------------------------------------------------------

\title{\normalfont\spacedallcaps{Brittle fracture in Charpy impact test}} % The article title
\subtitle{\normalfont\spacedallcaps{Influence of material parameters}}

%\subtitle{Subtitle} % Uncomment to display a subtitle

\author{\spacedlowsmallcaps{Jules Royer}} % The article author(s) - author affiliations need to be specified in the AUTHOR AFFILIATIONS block

\date{\today} % An optional date to appear under the author(s)

%----------------------------------------------------------------------------------------

\begin{document}

%----------------------------------------------------------------------------------------
%	HEADERS
%----------------------------------------------------------------------------------------

\renewcommand{\sectionmark}[1]{\markright{\spacedlowsmallcaps{#1}}} % The header for all pages (oneside) or for even pages (twoside)
%\renewcommand{\subsectionmark}[1]{\markright{\thesubsection~#1}} % Uncomment when using the twoside option - this modifies the header on odd pages
\lehead{\mbox{\llap{\small\thepage\kern1em\color{halfgray} \vline}\color{halfgray}\hspace{0.5em}\rightmark\hfil}} % The header style

\pagestyle{scrheadings} % Enable the headers specified in this block

%----------------------------------------------------------------------------------------
%	TABLE OF CONTENTS & LISTS OF FIGURES AND TABLES
%----------------------------------------------------------------------------------------

\maketitle % Print the title/author/date block

\setcounter{tocdepth}{2} % Set the depth of the table of contents to show sections and subsections only

\tableofcontents % Print the table of contents

\listoffigures % Print the list of figures

\listoftables % Print the list of tables

%----------------------------------------------------------------------------------------
%	ABSTRACT
%----------------------------------------------------------------------------------------

%----------------------------------------------------------------------------------------
%	AUTHOR AFFILIATIONS
%----------------------------------------------------------------------------------------

\let\thefootnote\relax\footnotetext{* \textit{Department of Biology, University of Examples, London, United Kingdom}}

\let\thefootnote\relax\footnotetext{\textsuperscript{1} \textit{Department of Chemistry, University of Examples, London, United Kingdom}}

%----------------------------------------------------------------------------------------

\newpage % Start the article content on the second page, remove this if you have a longer abstract that goes onto the second page

%----------------------------------------------------------------------------------------
%	INTRODUCTION
%----------------------------------------------------------------------------------------

\section{Introduction}
Goal of this work is to analyze the conditions under which fractures
and crack propagate in materials. It relies on the theory of fracture mechanics.
It has many important applications, especially in the industry field,
in which built mechanical components often have flaws.
The probability of the failure of the material during
its operation must be assessed.
To that end, engineers do a damage tolerance analysis.

The Charpy impact test is commonly used to study fractures.
Its functioning is quite simple: a striker is dropped and hits a
notched tensile, the difference between the final and initial heights
correspond to the energy dissipated in the specimen to create a fracture.

In this work, Finite Element Element simulations of this
experiment are done to analyze
the influence of some parameters of the material on its resistance
to fracture.

\begin{figure}[H]
    \centering
    \includegraphics[width=0.7\columnwidth]{mouton_charpy.png}
\end{figure}

\section{Preliminary Finite Element Model}

We present here a FEM model of the Charpy specimen.

It has an
isotropic elastic behavior under low strain or stress, with a
Young Modulus $E=208 \text{GPa}$ which is standard for steel.
It then has a isotropic non linear plastic behavior ruled by:
$ \sigma = R_0 + Q(1 - \exp^{-bp}) $ where $p$ is the cumulative
plastic deformation $\displaystyle p = \int_{t=0}^{t=t*}\varepsilon_{p}(t)dt$.

To model the transition in plastic regime, the \textit{Von Mises} criterion
is used. It relies on the \textit{Von Mises} equivalent stress:

\begin{equation}
    \sigma_{eq} = \sqrt{\frac{1}{2}\left((\sigma_1 - \sigma_2)^2+(\sigma_2 - \sigma_3)^2+(\sigma_3 - \sigma_1)^2\right)}
\end{equation}

Lastly, we use the Beremin model to predict the evolution of the crack.
Concerning the boundary conditions, anvils on which the specimen lies
are immobile, the striker has a prescribed vertical downward movement at
constant speed $\dot{u}_2 = -1 mm.s^{-1}$. The ligament, ie the nodes
in the vertical plane cutting the specimen in half, are also immobile
along the horizontal axis.
While this assumption is not physically justified, it is useful to
shrink the problem size to only the right half of all aforementioned
objects, because the problem can be considered symmetric along the ligament.


The first simulation returned an error during the last iterations.

\begin{figure}[ht]
    \centering
    \subfloat[Initial mesh]{\includegraphics[width=0.45\textwidth]{p1_init_mesh.png}}
    \hfill
    \subfloat[Vertical displacement]{\includegraphics[width=0.45\textwidth]{p1_u2_map.png}}
    \\
    \subfloat[\textit{Von Mises} stress]{\includegraphics[width=0.45\textwidth]{p1_sigmises_map.png}}
    \hfill
    \subfloat[cumulative plastic deformation]{\includegraphics[width=0.45\textwidth]{p1_epcum_map.png}}
    \caption{Results of experiment 1}
    \label{fig:res_exp_1}
\end{figure}

We observe a concentration of high \textit{Von Mises} stress
around the notch and near the striker, $\sigma_{eq} > 700 \text{MPa}$, and
also a bit of
plastic deformation.
A possible explanation of the error could be a lacking mesh precision
around those zones and also near the anvils where it reaches
$\sim 310 \text{MPa}$.

\section{Local Mesh Refinement}

Thanks to this analysis, we choose a finer mesh in those areas :

\begin{figure}[h]
    \centering
    \includegraphics[width=\textwidth]{p2_refine_props.png}
    \caption{Propositions of mesh refinement.}
\end{figure}

The $4$-th mesh fulfills our desires of refinement, so we select it.
In order to select the size of the smallest mesh elements, we rerun the
previous experiment with $40 \mu m, 10 \mu m, 4 \mu m$:

\begin{figure}[ht]
    \centering
    \subfloat[Refined mesh]{\includegraphics[width=0.45\textwidth]{p2_init_mesh.png}}
    \hfill
    \subfloat[Relation force on striker - displacement]{\includegraphics[width=0.45\textwidth]{p2_force_ud.png}\label{fig:force-displ}}
    \\
    \subfloat[Relation first principal stress - y coordinate]{\includegraphics[width=0.45\textwidth]{p2_sig1_y.png}\label{fig:firstprin-y}}
    \hfill
    \subfloat[Relation cumulative plastic deformation - y coordinate]{\includegraphics[width=0.45\textwidth]{p2_epcum_y.png}\label{fig:epcum-y}}
    \caption{Comparison of mesh refinements}
    \label{fig:res_exp_refine}
\end{figure}

With these finer mesh, the previous simulation now converges.
Results on Figures~\ref{fig:force-displ} and ~\ref{fig:firstprin-y} quite differ between $40 \mu m$
and the $2$ smaller sizes, meaning we gain some model accuracy with refinement.
However, $10 \mu m$ and $4 \mu m$ have similar results for our precision,
but the $10 \mu m$ simulation was $15$ times faster, therefore it is
more interesting for us to choose it instead of the smallest size.
We see again on Figures~\ref{fig:firstprin-y} and ~\ref{fig:epcum-y} the concentration
of stress and plastic deformation near the notch ie in the low $y$ zones.

\section{Geometric Parameter Sensitivity}

\begin{wrapfigure}{r}{0.5\textwidth}
    \centering
    \includegraphics[width=0.6\textwidth]{p3_geom_params.png}
    \caption{Considered geometric parameters.}
\end{wrapfigure}

Now that we have a more adapted mesh and converging simulations,
we wish to assess the dependency of the geometry of the Charpy specimen
in its probability of failure. Numerical simulation is here useful
to run multiple Charpy tests, that in reality are hard to set up.
The considered parameters are the notch radius $N_R$, the height $H$
and half of the width $H_W$:

The strategy is here to run $3$ simulations for each parameter,
in each of them we add a certain variation of the parameter: $-20\%$, $0\%$ and $+20\%$,
while keeping every other at its default value.

\begin{figure}[H]
    \centering
    \subfloat[Probability of failure for various $N_R$]{\includegraphics[width=0.45\textwidth]{p3_varia_NR.png}}
    \hfill
    \subfloat[Zoom]{\includegraphics[width=0.45\textwidth]{p3_varia_NR2.png}}
    \\
    \subfloat[Probability of failure for various $H$]{\includegraphics[width=0.45\textwidth]{p3_varia_H.png}}
    \hfill
    \subfloat[Zoom]{\includegraphics[width=0.45\textwidth]{p3_varia_H2.png}}
    \\
    \subfloat[Probability of failure for various $H_W$]{\includegraphics[width=0.45\textwidth]{p3_varia_Hw.png}}
    \hfill
    \subfloat[Zoom]{\includegraphics[width=0.45\textwidth]{p3_varia_Hw2.png}}

    \caption{Probability of failure with respect to the striker displacement for
        various values of $N_r$, $H$ and $H_W$.}
    \label{fig:param-var}
\end{figure}

On Figure~\ref{fig:param-var}, we see that the probability of failure curve
has a similar shape for each parameter. The plots of the right column are zooms
on the zones of highest variance between curves, with $U_d \in [0.5 \text{mm}, 0.7 \text{mm}]$.
On these zooms, all $P_f$ curves are approximately linear, with same slope
$\frac{\Delta P_f}{\Delta U_d} \sim \frac{72-48}{0.7-0.5} = 120 \%.mm^{-1}$.
Only the offsets at a given displacement $u_d$ differ between parameters variations $v_1,v_2 \in \{\pm 20\%, 0\%\}$ ie
$|P_{f,v_1}(u_d) - P_{f,v_2}(u_d)|$.
Variating $H_W$ by $\pm 20\%$ induces the smallest variation of $P_f$, of $\sim 1\%$,
where for $N_R$ and $H$, $\Delta P_f \sim 5\%$.
Therefore $H$ and $N_R$ are the 2 most impactful parameters on the failure probability.
The lower the notch radius and the higher the height, the higher the failure
probability. But if one industrial operator should choose to focus only
on one parameter, I would advice to focus on $H$ that is simpler to
adjust than $N_R$, eventhough the latter seem more dangerous, because
the failure probability increases as $N_R$ diminishes, which makes
the notch harder to spot.

\section{The Beremin Model}

A difficulty link to estimating the failure probability is that failure
sometimes appear eventhough the applied stress is below the yield value
of the material, because there were defects and cracks in it. The Beremin
model focus on those local micromechanisms.

The idea of the Beremin model relies on $2$ main assumption.
The first states that microcraks appear dut to inhomogeneous plastic
deformation in grains, the second models the propagation of those
micracks with the stress normal to their planes exceeding a
critical stress $\sigma_c$. This value is often approximated for
a microcrak of length $l_0$ by:
\begin{equation}
    \sigma_c = \sqrt{\frac{2E\gamma}{\pi(1-\nu^2)l_0}}
\end{equation}

The local approach is to discretize the stressed volume $V_p$ that
becomes plastic into $n$ smaller ones of volume $V_0$, mutually independant
and each containing a notch.
It is assumed the general stressed region fails if and
only if one of its subregion fails, and the failure will appear
at the longest microcrak.
Given the probability distribution of the crack within one small volume :$\mathbb{P}_{crack}$,
the probability of failure under stress $\sigma$ in this volume is the probability to find a
crack longer than a critical length $l_c$ linked to a critical stress:
\begin{equation}
    \mathbb{P}_{V_0,fail}(\sigma) = \displaystyle \int_{l_c}^{+\infty} \mathbb{P}_{crack}(l_0)d0 = \left(\frac{\sigma}{\sigma_u}\right)^m
\end{equation}

Where $m$ and $\sigma_u$ are constant. we then find the overall
probability of failure:
\begin{equation}
    P_{f} = 1 - \exp\left(-\left(\frac{\sigma_w}{\sigma_u}\right)^m\right)
\end{equation}

Where $\sigma_w = \displaystyle \left( \frac{1}{V_0} \int_{V_p} \sigma_{I,p}^m \right)^{\frac{1}{m}}$
represents a mean local stress among all subvolumes.

The $2$ Beremin parameters $m$ and $\sigma_u$ have specific influence on $P_f$:
$m$ is linked to the distribution of length of microcraks. Beremin originally
used $ \mathbb{P}_{crac}(l_0)dl_0 = \frac{\alpha}{l_0^{\beta}}$ with
$\beta = \frac{m}{2} + 1$. To assess its impact on $P_f$, we can
assume $\sigma_w$ is approximately constant when $m$ varies, then
an increase of $m$ increases $P_f$ if $\sigma_w > \sigma_u$ and
decreases it if $\sigma_w < \sigma_u$. For $\sigma_u$, its increase
induces a decrease of $P_f$.
Those relations are intuitive if we interpret $m$ as the number of
subzones of defects in the stressed zone, and $\sigma_u$ a stress linked
to the critical $\sigma_c$: increasing this threshold is equivalent
to say the material is more resistant to fracture, where increasing
the number of defects increases its instability, and the probability
of failure when the local stress exceeds the critical threshold.

When we look at the results of simulations on Figure~\ref{fig:var-m-sigu}, we see the trend that $m$
shifts the curve of $P_f$ to the right, without noticeably changing
the slope, probably because the variation of $\sigma_w$ w.r.t $m$ must
be considered. As expected, a $\sigma_u$ reduces the slope in the
linear part.

\begin{figure}[H]
    \centering
    \subfloat[Probability of failure for various $m$.]{\includegraphics[width=0.45\textwidth]{p4_pf_ud_m.png}}
    \hfill
    \subfloat[Probability of failure for various $\sigma_u$.]{\includegraphics[width=0.45\textwidth]{p4_pf_ud_sigu.png}}
    \label{fig:var-m-sigu}
\end{figure}


We wish now to find $m$ and $\sigma_u$ for a given material, for which
we experimentally statistics of failure $P_f$ for load $u_d$.
Because the FEM model has long computation time, a regression approach
will be very time consuming.
In fact, one simulation that computes
$u_d,P_f$ given $m,\sigma_u$ takes $\sim 1$ minute.
Therefore it is preferable to adjust manually those parametres, knowing
their influence as explained before. Pretty quickly a satisfying
estimation has been found on Figure~\ref{fig:estim-m-sigu}.

The mean square error of the estimation interpolated on the experimental
values of $u_d$ is $\displaystyle \sum_{u_{d,exp}}\left(P_{f,exp}(u) - P_{f,sim}(u)\right)^2 = 10.14$.

\begin{figure}[H]
    \centering
    \includegraphics[width=0.45\textwidth]{p4_m_sig_estim2.png}
    \caption{Observed $P_f(u_d)$ (blue dots). Estimated $P_f(u_d)$ with FEM method (blue curve) and \\ $m=15,\sigma_u=1830$.}
    \label{fig:estim-m-sigu}
\end{figure}

%----------------------------------------------------------------------------------------
%	METHODS
%----------------------------------------------------------------------------------------

\section{Methods}

\lipsum[5] % Dummy text

\begin{enumerate}[noitemsep] % [noitemsep] removes whitespace between the items for a compact look
    \item First item in a list
    \item Second item in a list
    \item Third item in a list
\end{enumerate}

%------------------------------------------------

\subsection{Paragraphs}

\lipsum[6] % Dummy text

\paragraph{Paragraph Description} \lipsum[7] % Dummy text

\paragraph{Different Paragraph Description} \lipsum[8] % Dummy text

%------------------------------------------------

\subsection{Math}

\lipsum[4] % Dummy text

\begin{equation}
    \cos^3 \theta =\frac{1}{4}\cos\theta+\frac{3}{4}\cos 3\theta
    \label{eq:refname2}
\end{equation}

\lipsum[5] % Dummy text

\begin{definition}[Gauss]
    To a mathematician it is obvious that
    $\int_{-\infty}^{+\infty}
        e^{-x^2}\,dx=\sqrt{\pi}$.
\end{definition}

\begin{theorem}[Pythagoras]
    The square of the hypotenuse (the side opposite the right angle) is equal to the sum of the squares of the other two sides.
\end{theorem}

\begin{proof}
    We have that $\log(1)^2 = 2\log(1)$.
    But we also have that $\log(-1)^2=\log(1)=0$.
    Then $2\log(-1)=0$, from which the proof.
\end{proof}

%----------------------------------------------------------------------------------------
%	RESULTS AND DISCUSSION
%----------------------------------------------------------------------------------------

\section{Results and Discussion}

Reference to Figure~\vref{fig:gallery}. % The \vref command specifies the location of the reference

\begin{figure}[tb]
    \centering
    \includegraphics[width=0.5\columnwidth]{GalleriaStampe}
    \caption[An example of a floating figure]{An example of a floating figure (a reproduction from the \emph{Gallery of prints}, M.~Escher,\index{Escher, M.~C.} from \url{http://www.mcescher.com/}).} % The text in the square bracket is the caption for the list of figures while the text in the curly brackets is the figure caption
    \label{fig:gallery}
\end{figure}

\lipsum[10] % Dummy text

%------------------------------------------------

\subsection{Subsection}

\lipsum[11] % Dummy text

\subsubsection{Subsubsection}

\lipsum[12] % Dummy text

\begin{description}
    \item[Word] Definition
    \item[Concept] Explanation
    \item[Idea] Text
\end{description}

\lipsum[12] % Dummy text

\begin{itemize}[noitemsep] % [noitemsep] removes whitespace between the items for a compact look
    \item First item in a list
    \item Second item in a list
    \item Third item in a list
\end{itemize}

\subsubsection{Table}

\lipsum[13] % Dummy text

\begin{table}[hbt]
    \caption{Table of Grades}
    \centering
    \begin{tabular}{llr}
        \toprule
        \multicolumn{2}{c}{Name}       \\
        \cmidrule(r){1-2}
        First name & Last Name & Grade \\
        \midrule
        John       & Doe       & $7.5$ \\
        Richard    & Miles     & $2$   \\
        \bottomrule
    \end{tabular}
    \label{tab:label}
\end{table}

Reference to Table~\vref{tab:label}. % The \vref command specifies the location of the reference

%------------------------------------------------

\subsection{Figure Composed of Subfigures}

Reference the figure composed of multiple subfigures as Figure~\vref{fig:esempio}. Reference one of the subfigures as Figure~\vref{fig:ipsum}. % The \vref command specifies the location of the reference

\lipsum[15-18] % Dummy text

\begin{figure}[tb]
    \centering
    \subfloat[A city market.]{\includegraphics[width=.45\columnwidth]{Lorem}} \quad
    \subfloat[Forest landscape.]{\includegraphics[width=.45\columnwidth]{Ipsum}\label{fig:ipsum}} \\
    \subfloat[Mountain landscape.]{\includegraphics[width=.45\columnwidth]{Dolor}} \quad
    \subfloat[A tile decoration.]{\includegraphics[width=.45\columnwidth]{Sit}}
    \caption[A number of pictures.]{A number of pictures with no common theme.} % The text in the square bracket is the caption for the list of figures while the text in the curly brackets is the figure caption
    \label{fig:esempio}
\end{figure}

%----------------------------------------------------------------------------------------
%	BIBLIOGRAPHY
%----------------------------------------------------------------------------------------

\renewcommand{\refname}{\spacedlowsmallcaps{References}} % For modifying the bibliography heading

\bibliographystyle{unsrt}

\bibliography{sample.bib} % The file containing the bibliography

%----------------------------------------------------------------------------------------

\end{document}